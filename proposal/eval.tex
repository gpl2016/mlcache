\section{Evaluation}

The ideal applications that would benefit from the approach proposed in this document
are long-running processes with nondeterministic access patterns. In such cases,
a typical LRU caching strategy might not be optimal.

We plan to evaluate the system by comparing the performance of applications running
on top of it with the performance of the same application running on an unmodified
kernel. Examples of such applications are:

\begin{itemize}
\item A database server: depending on the access patterns, a better cache policy
is likely to produce measurable performance improvements. A good example candidate
for this kind of test is to run PostgreSQL with different datasets and under different
kinds of load.

\item A backup utility: tools like \texttt{dump(1)} and \texttt{restore(1)} typically
have to scan the entire file system when generating incremental backups. They are also
likely to be sensitive to modifications at the buffer cache layer.

\item A process that does sequential reading: we plan to measure the impact of our
changes on tools like \texttt{grep(1)}.
\end{itemize}
