\section{Related Work}

Previous work on PGo \cite{Zhang:2016} demonstrated the viability of translation from PlusCal to Go programs running on a single machine. PGoNet aims to extend this effort by compiling the resulting Go programs as distributed programs running on multiple machines. As scalability is no longer an afterthought for applications \cite{Cavage:2013}, this approach provides a promising methodology for application developers by marrying the simplicity of model checking provided by the PlusCal toolset and the performance as well as correctness of the translated distributed Go programs.

Other works have been developed to provide correct model checking of specifications and the corresponding implementations. Verdi is a framework for implementing fault-tolerant distributed systems \cite{Wilcox:2015}. The Verdi framework does not have a model checker so significant developer effort is required on providing correctness proofs as accompanying Coq programs. IronFleet, similarly, requires significant developer involvement to provide proofs as Dafny programs \cite{Hawblitzel:2015}. On the other hand, MODIST attempts to check the implementations rather than indirectly checking the specifications \cite{Yang:2009}. This method reduces the effort on the programmer but puts a prohibitively large burden on the checker due to state space explosion.
